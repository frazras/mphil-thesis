\chapter{Background and Motivation} % (fold)
\label{cha:background_and_motivation}
%\lipsum[1]~\parencite{modcom}

Robotics development is a continually growing field with a promising future. the
current state of robotics can be compared to that of the PC industry in its
embryonic years~\parencite{billgates}. Many of the issues that the PC industry faced are redefining
themselves in today's robotics arena. The most popular ones include high costs, a lack
of common standards and generic platforms. Whenever it becomes necessary to
develop a new robot for a particular task, it is often necessary to build
it from scratch. This is because there are few general purpose platforms that meet
all the needs of the typical robotics engineer, student and hobbyist~\parencite{r2p}. The
``killer app'' for the robotics industry will be a platform that can address these issues
and thereby create a broader adoption of robotics technology. As with the
computer industry, one of the main markets that utilize robotics is education.
Robotics is used to teach students of all ages and levels on topics ranging from
science to social skills~\parencite{kramer}. General Purpose Robotic kits are one of the many
`edutainment' methods that are developed. These kits provide the user with the
ability to build robots for a large range of applications.

An increased interest in robotics and robotics development has fuelled 
the search for a robotics platform that will be useful for schools and the 
typical hobbyist. There has been research into finding a suitable
platform that is flexible, inexpensive, extensible and capable of handling
projects of varying complexities~\parencite{bot-mate}. Popular kits exist that meet some of
these characteristics. For instance, the Lego Mindstorms\texttrademark  robotics
kit is one of the most popular robotics platforms that can achieve many of these
requirements. It however achieves a few of them through unconventional
`hacking' of the platform, or by purchasing third party equipment. This is not
always easy to accomplish for novice developers. 

``Studies show that robotics generates a high degree of student interest and engagement and
promotes interest in math and science careers .  The robotics platform also promotes learning 
of scientific and mathematic principles, through experimentation''~\parencite{barker}.
The  inclusion of robotics in a school curriculum has been
shown to improve performance in STEM subjects~\parencite{school}. 
Robotics development enables students to apply practical techniques to complete tasks and 
to solve scientific problems. The result or response from their work is
immediately observable through the physical response of the robot. In addition,
robotics  presents a `fun factor' that attracts young people.

\section{Similar Projects} % (fold)
\label{sec:similar}

%Mention in the field kits like Tetrixx
There are numerous robotics platforms, both custom built and general purpose that implement useful design strategies. The following outlines many of the core concepts found in these projects and the advantages of their approach.

The Tetrixx robotics project was developed in the late 1990's. It sought to overcome the issue of expandability being faced by its contemporaries such as MIT's 6.270 board and Handyboard~\parencite{handy} projects. To overcome this limitation, the Tetrixx project employed specialised sensor and actuator expansion boards connected over a bus. The bus supported up to 64 expansion boards and allowed the Tetrixx kit to manage far more peripherals than supported by the connectors from one single controller~\parencite{tetrixx}.

Another interesting robotics kit project is the LEGO Mindstorms NXT robotics development platform. This is a very popular robotics kit among schools and hobbyists. It has three actuator ports and four sensor ports on its controller unit. Additional sensors and actuators can be added either by pairing with another controller or by purchasing third party multiplexing devices that allow multiple peripheral devices connected to one port. The port design is quite extensible with numerous analog and digital I/O pins and an \iic bus to support even very complex devices. The Mindstorms kit has the distinct advantage of having a very active community of developers who have worked to create a large set of programming interfaces that can be used to write robotics applications for the platform. These include compilers for some of the most popular programming languages such as, C, Java, Python, Ruby, Ada, Lua and Matlab. The LEGO Mindstorms kit uses a very modern architecture and is as a standard by which many robotics kits are measured. The LEGO Mindstorms kit however lacks ease of scalability some users require. It is also among one of the few commercial robotic kits that is open sourced. 

The architecture of the Hannibal Hexapod robotics project has a commendable capability of managing the input and output signals for over one hundred (100) physical sensors and actuators~\parencite{hannibal}. The Hexapod robot controller is based on the Subsumption Architecture and is fully distributed across eight(8) on-board computers: one each for the main controller, the robots body and each of its six legs. All these modules are connected via an \iic serial communication bus. These components were the realisation of the Hannibal Hexapod's design requirements of being scalable, modular, flexible, robust and adaptive.

The Rapid Robot Prototyping(R2P) project design goals were to be an inexpensive, open source, modular architecture for rapid prototyping of robotic applications~\parencite{r2p}. This architecture is aimed at students and hobbyists creating robotic applications on low powered microcontrollers. The architecture is built around a distributed system of modules communicating over a high speed serial data CAN bus. It emphasises its plug and play capability by providing ports on each module to support daisy-chaining of multiple peripheral boards. On the software aspect, the R2P has embedded firmware on its peripheral modules to act as a Hardware Abstraction Layer(HAL) to encourage modularity. On the main controller, it implemented a Real Time Operating System(RTOS), a publish/subscribe \hyperref[ssub:middleware]{middleware} and a Virtual Machine that supports an embedded scripting language. The CAN serial bus was used for inter-module communication. 
%
%

%{TABLE HERE}
%Table showing comparison of robotic platforms and a measure of their
%capabilities


\subsection{Modularity} % (fold)
\label{sub:modularity}
Modularity in the design of a robotics kit architecture defines the capability to incrementally add fully independent sub-systems to a robotics project without significantly modifying the existing configuration. According to Oraw and Tinder~\parencite{mars}, ``The modularity of a robot is demonstrated by its expandable intelligence. Sensory modules that implement the robot's data protocol can be plugged into the system without reprogramming the original components''. They also mention that the key to modularity is the use of peripheral modules. 

Many robotics projects opt to use a distributed system architecture. This enables the control of the overall system to be shared across multiple independently operating modules. These modules operate efficiently because there is very minimal usage of shared resources, this is because each module is usually self sufficient and simply passes information back and forth between itself and the main controller over the data bus. In a centralised system which is dependent on one main processor, management of peripherals can become complex when handling a large number of modules. The controller's resources and connections are finite and there may be contention for these resources. This can prevent the inclusion of additional modules without the need to remove a previous module or somehow hack the system.  Using a distributed system not only reduces complexity but it also allows for changes to be made across the entire architecture without affecting the existing setup~\parencite{avcithesis}.

A subsequent advantage of a modular architecture is software and hardware reuse, and rapid prototyping. Peripherals can be incrementally added to build more complex projects and subsequently reduce the overall cost of development~\parencite{modcom}. Different functions of the system can be delegated to independent and specialised modules. The aim was to simplify the control problem by decomposing the global task into local tasks for the robot's subsystems~\parencite{rdk}. This reduces the computational load on the main processor by delegating the majority of the low level sensor and actuator processing to sub-modules. 

Another common practice that many robotics architectures employ is the inclusion of peripheral sensor and actuator boards in their design. This can be seen in the Tetrixx project~\parencite{tetrixx} and numerous other robots built around a distributed processing architecture. Peripheral boards are secondary computational units that are responsible for low level control operations. These peripheral boards interpret and execute data and control commandscommunicated between the main controller and the board's sensors and actuators. They are connected to a peripheral bus which will be able to uniquely identify each module along the communication line. 

 
% subsection modularity (end)

\subsection{Scalability} % (fold)
\label{sub:scalability}
Robotics kits that can support projects of varying complexities should have the ability to support as many sensors and actuators as may be required for a complex robotics project. A robotics project can range from robots that complete simple tasks that require one or two sensors and actuators to robots that manage tens of sensors and actuators necessary to complete their task. In general, as the complexity of a robotics project increases, so does the number of peripherals required to complete the tasks. Projects that may solve reasonably technical tasks will sometimes have to improvise by finding creative ways to use available resources. However, there may be times that the scale of the project requires a large number of sensors and actuators to adequately perform the task. Robots that are known for using a large number of sensors and actuators are usually ones that try to replicate animal or human ability. One such example is the Hannibal Hexapod Robot. Hannibal receives a tremendous amount of sensor  information while continually controlling its almost 20 servo motors.  The spider-like robot has over 60 sensors of different types~\parencite{hannibal}. Other types of robots that utilize many sensors and actuators include robotics arms, full humanoid robots, and animal replicas like MARS: the Modular  Autonomous Robotics Snake~\parencite{mars}. In all of the above examples, the supporting architecture facilitated the inclusion and efficient management of a significant number of sensors and actuators.

 A distributed robotics architecture is a major contributor in decreasing the cost-to-scale factor. In a distributed system, the processing is shared between a central processing unit or a motherboard and numerous specialised modules. These modules are usually focused on doing simple tasks and therefore will not require complex hardware or devices to accomplish them. This can therefore allow them to to be inexpensive and simply connect to the motherboard to report the results of their computations and fetch data on request.

 Proper power management is very important when managing tens to hundreds of sensors and actuators. These devices can behave erroneously or simply fail to perform if they are not supplied with adequate power. It is also very important to place electrically noisy and high powered devices on separate power sources. This is especially true with motors and sensitive, low powered electronics should never share the same power source. This can electrically damage sensitive components of a controller and render the robot unusable. It is also necessary that the system be sized for its maximum power requirements and a capable power source selected. Under-powering a robot can also cause unpredictable results. Sometimes it may even be necessary to have multiple power sources on a robot that are adequately isolated. 

Another requirement that may be easily overlooked is a structured and organised wiring and connection system. A robotics project can become increasingly difficult to develop if there is a web of wires and connections attached to the dozens of devices that may be present on a robot. A design that will mitigate this issue is the use of a distributed system supported by a main bus with an addressable serial data connection. If the architecture is designed with a single peripheral bus as its communication backbone which connects all devices, most of the wiring can be placed in a single organised package. The advantage of using an addressable serial bus is that these usually utilize very few connection lines. They also allow for new modules to be daisy chained onto each other. Any addition of a new set of sensors or actuators will simply equate to physically plugging a single set of wires from one module to another.

A very popular communication interface encountered in multiple projects was the Inter-Integrated Circuit (\iic) bus. \iic is a multi-master, addressable slave, two wire bus serial protocol. It supports up to 112 uniquely addressable devices per bus (128 minus 16 reserved addresses), 8-bit oriented, bidirectional data transfers can be made at 100 Kbits/s in the Standard-mode, 400 Kbits/s in the Fast-mode, 1 Mbits/s in Fast-mode Plus, or up to 3.4 Mbits/s in the High-speed mode ~\parencite{edubots, i2cfaq}. While operating in its standard-mode, the I2C bus can be as long as 9 - 12 feet without any significant noise interference ~\parencite{hannibal}. This distance is quite sufficient for a robotics kit where the peripherals are not usually more than 3 feet away from the controller circuit. As such, \iic provides a low-cost solution for interconnecting large numbers of devices operating at relatively slow speeds, such as sensors or other external devices connected to a microcontroller. \iic inherently supports collision detection and bus arbitration, bus arbitration refers to a portion of the protocol that ensures that when multiple masters try to control the bus simultaneously, \iic allows only one master full control of the bus while queuing the prospective master without any corruption or data loss~\parencite{nxpi2c}.  A number of robotics projects implement this method of data communication, including the popular NXT LEGO Mindstorms Kit, ISocRob, and the Hannibal Hexapod autonomous robot~\parencite{hannibal, Ventura}.

%Scalable: Support for multiple sensors and actuators.



\subsection{Extensibility} % (fold)
\label{sub:extensibility}
A truly extensible robotics system permits future addition and support of new sensors and actuators that might not have been included in the initial design. Extensibility of a robotics kit provides developers with the freedom to define the hardware configuration with respect to the available types of sensors and actuators~\parencite{rdk}. A robotics kit may never be able to guarantee support for all types of sensors and actuators, but a truly extensible architecture will provide the means for developers to configure support for their particular hardware. 

\subsubsection{Middleware} % (fold)
	\label{ssub:middleware}
	Bakken et al.~\parencite{bakken2001middleware} defined middleware as follows: ``a class of software technologies designed to help manage the complexity and heterogeneity inherent in distributed systems. It is defined as a layer of software above the operating system but below the application program that provides a common programming abstraction across a distributed system.'' Middleware provides abstraction and seamless communication between modules in distributed robotics system. The middleware handles communication between devices without exposing the details of the protocol or its requirements and the details of the systems hardware on either end. The middleware is used to determine what type of communication medium and protocol is appropriate and creates a standard platform independent interface for interaction. In other words, the middleware just simply presents a standard interface that communicates data and instructions.
	In a review of popular robotics middleware, Kramer~\parencite{kramer} cites some major advantages of using middleware. He states that it aids in software modularity. The middleware allows for pluggable libraries that are ignorant of the actual hardware implementations. This accomplishes the platform independence and portability, Kramer found these to be common attributes of most robotics middleware.
	

%Extensible: Permit Customized commands for peripherals



% subsection extensibility (end)

\subsection{Economics} % (fold)
An important requirement that will permit wider adoption of robotics kits is ensuring that the kit's cost and its cost to scale are kept relatively low. The cost of electronics is a decreasing function of time, however these prices are not always instantly obvious in the field of robotics. It is almost an accepted fact that robotics is an expensive field, the prices of robotics kits range from hundreds to thousands of U.S. dollars. This is a major point of restriction, robots are expensive and often beyond the resource of many persons and even universities and organisations who want to delve into the field of robotics~\parencite{vr}. In order to make a robotics kit that will be an attractive option for hobbyist roboticists and school programmes even as low as the elementary school level, it has to be available at a competitive price point. A very popular platform for doing robotics is the Arduino development board which at the time of this publication ranged from U.S. \$15 USD to \$70 USD. The prices are based on type of Arduino, place of purchase and whether the Arduino is a genuine model or a clone. The Arduino development board is not specialised for robotics but has all the major tools necessary to build a functional robot if the developer is skilled in software development and electronics. The Arduino is also one of the most popular development platforms at robotics competitions.
The Arduino board also has a distinct advantage of being open-source. This gives the developer the advantage of having access to free software upgrades and longer hardware support. Being open-source also means that a skilled developer will be able to make modification to the system's code and even build their own hardware replica. These advantages can mean significant savings if a solution can be accomplished by the developer instead of purchasing an upgraded version or proprietary hardware.

A comparable system that is currently used many in schools and by hobbyist is the Lego Mindstorms robotics kit. The controller for the LEGO Mindstorms kit without sensors and actuators starts at \$250. 

%Inexpensive: Relatively low initial cost andd cost to scale
\label{sub:economics}

% subsection economics (end)

% section design_objectives (end)